\documentclass{article}

\usepackage{comment}

\begin{comment}
% w: create table by hand

\usepackage{booktabs}

\begin{document}

\begin{table}[h!]
  \centering
  \caption{Table for trial}
  \label{tab:table1}
  \begin{tabular}{lll}                    % l c r align content to left, center, right
    \toprule
    name & gender & age\\
    \midrule
    Alice & F & 26\\
    Ben & M & 19\\
    Chris & M & 35\\
    \bottomrule
  \end{tabular}
\end{table}

\end{document}

\end{comment}

%--------------------------------------------------------------
% w: create table from csv file
% w: type "texdoc siunitx" in cmd command line to open document for the package

% \documentclass{article}

\usepackage{booktabs} % For \toprule, \midrule and \bottomrule
\usepackage{siunitx} % Formats the units and values
\usepackage{pgfplotstable} % Generates table from .csv

% Setup siunitx:
\sisetup{
  round-mode          = places, % Rounds numbers
  round-precision     = 2, % to 2 places
}

\begin{document}

\begin{table}[h!]
  \begin{center}
    \caption{Autogenerated table from .csv file.}
    \label{table1}
    \pgfplotstabletypeset[
      multicolumn names, % allows to have multicolumn names
      col sep=comma, % the seperator in our .csv file
      display columns/0/.style={
		column name=$Value\,1$, % name of first column
		column type={S},string type},  % use siunitx for formatting
      display columns/1/.style={
		column name=$Value\,2$,
		column type={S},string type},
      every head row/.style={
		before row={\toprule}, % have a rule at top
		after row={
			% \si{\ampere} & \si{\volt}\\    % w: will display V and A, no need here 
			\midrule} % rule under units
			},
		every last row/.style={after row=\bottomrule}, % rule at bottom
    ]{testcsvtable.csv} % filename/path to file
  \end{center}
\end{table}

\end{document}